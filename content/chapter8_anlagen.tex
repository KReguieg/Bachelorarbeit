\appendix
\chapter{Anlagen}

\section{GazeManager.cs}
\begin{lstlisting}[caption=GazeManager.cs, label=code:GazeManager]
using UnityEngine;

public class GazeManager : Singleton<GazeManager> {

  /// Declaration of variables, setters and getters.
  [...]

  private void Update() {
	gazeOrigin = Camera.main.transform.position;
	gazeDirection = Camera.main.transform.forward;
	gazeStabilizer.UpdateHeadStability(gazeOrigin, Camera.main.transform.rotation);
	gazeOrigin = gazeStabilizer.StableHeadPosition;
	UpdateRaycast();
  }

  /// <summary>
  /// Calculates the Raycast hit position and normal.
  /// </summary>         
  private void UpdateRaycast() {
	RaycastHit hitInfo;
	Hit = Physics.Raycast(gazeOrigin, gazeDirection, out hitInfo, MaxGazeDistance, RaycastLayerMask);
	HitInfo = hitInfo;
	if (Hit) { // If raycast hit a hologram...
	  Position = hitInfo.point;
	  Normal = hitInfo.normal;
	}
	[...]
  }
}
\end{lstlisting}

\section{CursorManager.cs}
\begin{lstlisting}[caption=CursorManager.cs, label=code:CursorManager]
using UnityEngine;

/// <summary>
/// CursorManager class takes Cursor GameObjects.
/// One that is on Holograms and another off Holograms.
/// Shows the appropriate Cursor when a Hologram is hit.
/// Places the appropriate Cursor at the hit position.
/// Matches the Cursor normal to the hit surface.
/// </summary>
public class CursorManager : Singleton<CursorManager> {

  [Tooltip("Drag the Cursor object to show when it hits a hologram.")]
  public GameObject CursorOnHolograms;
	
  [Tooltip("Drag the Cursor object to show when it does not hit a hologram.")]
  public GameObject CursorOffHolograms;
	
  void Awake() {
	if (CursorOnHolograms == null || CursorOffHolograms == null) {
	  return;
	}
	
	// Hide the Cursors to begin with.
	CursorOnHolograms.SetActive(false);
	CursorOffHolograms.SetActive(false);
  }
	
  void Update() {
	if (GazeManager.Instance == null || CursorOnHolograms == null || CursorOffHolograms == null) {
	  return;
	}
	
	if (GazeManager.Instance.Hit) {
 	  CursorOnHolograms.SetActive(true);
 	  CursorOffHolograms.SetActive(false);
	} else {
	  CursorOffHolograms.SetActive(true);
	  CursorOnHolograms.SetActive(false);
	}
	
	gameObject.transform.position = GazeManager.Instance.Position;	
	gameObject.transform.up = GazeManager.Instance.Normal;
  }
}
\end{lstlisting}

\section{InteractibleManager.cs}
\begin{lstlisting}[caption=InteractibleManager.cs, label=code:InteractibleManager]
using UnityEngine;

/// <summary>
/// InteractibleManager keeps tracks of which GameObject
/// is currently in focus.
/// </summary>
public class InteractibleManager : Singleton<InteractibleManager> {

  public GameObject FocusedGameObject { get; private set; }
  private GameObject oldFocusedGameObject = null;

  void Start() {
	FocusedGameObject = null;
  }

  void Update() {
	oldFocusedGameObject = FocusedGameObject;

	if (GazeManager.Instance.Hit) {
	  RaycastHit hitInfo = GazeManager.Instance.HitInfo;
	  if (hitInfo.collider != null) {
		///Assign the hitInfo's collider gameObject to the FocusedGameObject.
		FocusedGameObject = hitInfo.collider.gameObject;
	  } else {
		FocusedGameObject = null;
	  }
	} else {
		FocusedGameObject = null;
	}

	if (FocusedGameObject != oldFocusedGameObject) {
	  ResetFocusedInteractible();

	  if (FocusedGameObject != null) {
		if (FocusedGameObject.GetComponent<Interactible>() != null) {
		  // Send a GazeEntered message to the FocusedGameObject.
		  FocusedGameObject.SendMessage("GazeEntered");
		}
	  }
	}
  }

  private void ResetFocusedInteractible() {
	if (oldFocusedGameObject != null) {
	  if (oldFocusedGameObject.GetComponent<Interactible>() != null) {
	  	///Send a GazeExited message to the oldFocusedGameObject.
		oldFocusedGameObject.SendMessage("GazeExited");
  	  }
  	}
  }
}
\end{lstlisting}
\section{GestureAction.cs}
\begin{lstlisting}[caption=GestureAction.cs, label=code:GestureAction]
using System;
using UnityEngine;
using MonoBrick.EV3;
/// <summary>
/// GestureAction performs custom actions based on
/// which gesture is being performed.
/// </summary>
public class GestureAction : MonoBehaviour {
  [Tooltip("Translation max speed controls amount of translation.")]        
  public float TranslationSensitivity = 10.0f;  
    
  private Vector3 manipulationPreviousPosition;        
  private float translationFactor;     
  private EV3Manager ev3;     
  PositionManager positionManager;     
 
  void Start() {         
	ev3 = GameObject.Find("Managers").GetComponent<EV3Manager>();        
	Debug.Log(ev3 + "loaded!\nGetting in ready position...");
	ev3.Beep();         
	ev3.MotorBlocked = false;         
	ev3.MoveRobotArmToStartingPosition();         
	positionManager = GetComponent<PositionManager>();     
  }     
  
  void Update(){      
    PerformTranslation();         
    PerformGrab();     
  }     
  
  private void PerformTranslation() {         
    if (GestureManager.Instance.IsNavigating && HandsManager.Instance.FocusedGameObject == gameObject) {             
      // Calculate translationFactor based on GestureManager's NavigationPosition.X and multiply by TranslationSensitivity.             
      // This will help control the amount of translation.             translationFactor = GestureManager.Instance.NavigationPosition.x;          
    
      float clampTranslation = Mathf.Clamp(translationFactor, -1f, 1f);      
      positionManager.SetPosition(clampTranslation / 2 + 0.5f);
      ev3.MoveRobot((sbyte)(clampTranslation * 10));         
    } else {             
	  positionManager.SetPosition(0.5f);             
	  ev3.StopRobot();         
	}     
  }     
  
  private void PerformGrab() {         
	if (GestureManager.Instance.IsNavigating && HandsManager.Instance.FocusedGameObject == gameObject) {             
	  // Calculate translationFactor based on GestureManager's NavigationPosition.X and multiply by TranslationSensitivity.             
	  // This will help control the amount of translation.             translationFactor = GestureManager.Instance.NavigationPosition.z;         
	  transform.Translate(new Vector3(0, 0, translationFactor / TranslationSensitivity));             
	  transform.position = new Vector3(transform.position.x, transform.position.y, Mathf.Clamp(transform.position.z , - 1f, 1f));
	}
  }
}
\end{lstlisting}
\section{VoiceManager.cs}
\begin{lstlisting}[caption=VoiceManager.cs, label=code:VoiceManager]
using System.Collections;
using System.Collections.Generic;
using UnityEngine;
using UnityEngine.Windows.Speech;
using System.Linq;
using System;

public class VoiceManager : MonoBehaviour {
  KeywordRecognizer keywordRecognizer;
  Dictionary<string, System.Action> keywords = new Dictionary<string, System.Action>();
  private EV3Manager ev3;
  
  // Use this for initialization
  void Start () {         
	ev3 = GetComponent<EV3Manager>();         
	keywords.Add("Grab", () => {             
	  ev3.GrabObject();         
	});         
	
	keywords.Add("Place", () => {             
	  ev3.PlaceObject();         
	});         
	
	keywords.Add("Stop", () => {             
	  ev3.Stop();         
	});
	
	keywordRecognizer = new KeywordRecognizer(keywords.Keys.ToArray());         keywordRecognizer.OnPhraseRecognized += KeywordRecognizer_OnPhraseRecognized;         keywordRecognizer.Start();
  }     
  
  private void KeywordRecognizer_OnPhraseRecognized(PhraseRecognizedEventArgs args) {
	Debug.Log("Keyword Recognized..."); 
	System.Action keywordAction; 
	if (keywords.TryGetValue(args.text, out keywordAction)) { 
	  keywordAction.Invoke(); 
	}
  }
  
  // Update is called once per frame
  void Update () {
  
  } 
}
\end{lstlisting}
\section{EV3Manager.cs}
\begin{lstlisting}[caption=EV3Manager.cs, label=code:EV3Manager]
using System.Collections;
using System.Collections.Generic;
using UnityEngine; using MonoBrick.EV3;
using System; using System.Threading;

public class EV3Manager : MonoBehaviour {
  public Brick<TouchSensor, Sensor, Sensor, Sensor> ev3;     
  private bool motorBlocked;   
    
  public bool MotorBlocked {         
	get { return motorBlocked; }
	set { motorBlocked = value;}
  }
  
  // Use this for initialization
  void Start () {         
	ev3 = new Brick<TouchSensor, Sensor, Sensor, Sensor>("WiFi");         
	ev3.Connection.Open();
	ev3.Sensor1.Mode = TouchMode.Boolean;         
	ev3.MotorA.ResetTacho();         
	ev3.MotorB.ResetTacho();         
	ev3.MotorC.ResetTacho();         
	MotorBlocked = false;
  }
  
  // Update is called once per frame
  void Update () { 
    Debug.Log(ev3.Sensor1.Read());         
    Debug.Log("Motor Blocked?= " + motorBlocked);         
    Debug.Log("TACHO B= " + ev3.MotorB.GetTachoCount());         
    Debug.Log("TACHO C= " + ev3.MotorC.GetTachoCount()); 
            
    if (ev3.Sensor1.Read() == 1 && MotorBlocked == true) { 
	  ev3.MotorB.Brake();             
	  MotorBlocked = false;
	}
  }     
  
  public void MoveRobot(sbyte speed) {
	ev3.MotorA.On(speed);     
  }     
  
  public void StopRobot() {         
	ev3.MotorA.Off();     
  }     
  
  public void MoveRobotArmToStartingPosition() {
	ev3.MotorB.On(-25);         
	MotorBlocked = true;     
  }     
  
  public void Beep() { 
	ev3.Beep(10, 500);     
  }    
  
  public void GrabObject() { 
	WaitForMotorToStop();         
	ev3.MotorB.Off();         
	WaitForMotorToStop();         
	ev3.MotorC.MoveTo(10, 80, true);         
	WaitForMotorToStop();         
	ev3.MotorB.On(-25);         
	MotorBlocked = true;     
  }
  
  internal void PlaceObject() { 
	MotorBlocked = false;         
	ev3.MotorB.Off();         
	WaitForMotorToStop();         
	ev3.MotorC.MoveTo(25, 0, true);         
	WaitForMotorToStop();         
	ev3.MotorB.On(-25);         
	MotorBlocked = true;     
  }     
  
  void OnDestroy() {
	ev3.MotorA.Off();         
	ev3.MotorB.Off();         
	ev3.MotorC.Off();     
  }     
  
  void WaitForMotorToStop() {         
	Thread.Sleep(500);         
	while (ev3.MotorA.IsRunning() || ev3.MotorB.IsRunning() || ev3.MotorC.IsRunning())             
	  Thread.Sleep(500);     
  }
  
  public int GetSpeed() {         
  	return ev3.MotorA.GetSpeed();
  }
\end{lstlisting}
\section{Usability Test Transskript}\label{transscrikpt}
\paragraph*{Tester1, männlich, keine Vorkenntnisse, Officemanagement}
\begin{itemize}
	\item deutlicher machen wie man Roboter bewegt
\end{itemize}
\paragraph*{Tester2, weiblich, keine Vorkenntnisse, Officemanagement}
\begin{itemize}
	\item Nicht Muttersprachler
	\item Hart die Ready-Gesture zu verstehen.
	\item Augen/Kopf-Hand-Koordination schnellere Ausführung der Gesten nötig Frustration, weil die Gesten nicht Funktionieren.
	\item Konnte \frqq Grab\flqq\-Aktion nicht ausführen
\item Fehlfunktion schwierig/Motorblockade behindert
\end{itemize}
\paragraph*{Tester3, männlich, keine Vorkenntnisse, Gestaltung}
\begin{itemize}
	\item Hat versucht Cursor mit Auge oder Finger zu lenken
	\item Aussage: \frqq Funktioniert erstaunlich gut.\flqq
	\item Gaze schwierig
	\item Wenig Latenz!
	\item Aussage: \frqq Wie Magie!\flqq
\end{itemize}
\paragraph*{Tester4, weiblich, keine Vorkenntnisse, Gestaltung}
\begin{itemize}
	\item Icon-Feedback fürs Tippen gewünscht
	\item Frustum für Handdetection anzeigen
	\item Cyan blau gut
	\item Reagiert schnell
	\item Nur die wichtigsten Dinge auf Interface
\end{itemize}
\paragraph*{Tester5, männlich, keine Vorkenntnisse, Projektmanagement}
\begin{itemize}
	\item Anwendung hat direkt funktioniert
	\item Es hat Spaß gemacht.
	\item Erstaunlich das Roboter sich bewegt hat.
	\item Hat Sprachbefehl zweimal sprechen müssen
	\item Nicht intuitivste Steuerung
	\item Absolute Werte Positionswerte
\end{itemize}
\paragraph*{Tester6, weiblich, keine Vorkenntnisse, Projektmanagement}
\begin{itemize}
	\item Das Holointerface ist ansprechend
	\item Gut, dass die Befehle nochmals  angezeigt werden
	\item Es hat Spaß gemacht.
	\item Leider Stimme nicht sehr gut erkannt
\end{itemize}
\paragraph*{Tester7, weiblich, keine Vorkenntnisse, Entwicklung}
\begin{itemize}
	\item Motorblockade nervt
	\item \frqq Tap and Hold\flqq\ zunächst nicht verstanden
	\item Wenn Gesten verstanden dann ist die Bedienung total leicht. Das gefällt
	\item Microsoft Gesten an für sich schwierig und ungewohnt
	\item Aufgabe in Interface Schritt für Schritt anzeigen
	\item Gut, dass die Befehle nochmals  angezeigt werden
\end{itemize}
\paragraph*{Tester8, männlich, keine Vorkenntnisse, Entwicklung}
\begin{itemize}
	\item Interaktionsobjekt auf Höhe des Holointerface
	\item Einfaches, gut verständliches Interface
	\item Alles wichtige in einem Blick!
	\item Der Aufgabe angemessen
	\item Kontrollelement zu tief, dadurch Unsicherheit beim Bedienen
	\item Aufgabe in Interface Schritt für Schritt anzeigen
	\item Gut, dass die Befehle nochmals  angezeigt werden
\end{itemize}