\chapter{Usability Test}\label{chapter:UsabilityTest}
\section{Think Aloud Test}\label{sec:ThinkAloudTest} 
Der \textit{Think Aloud}-Test, manchmal auch \textit{Thinking Aloud}-Test genannt, ist ein simples \textit{Usability}-Testverfahren, welches die Testprobanden dazu anhält: \frqq das System zu benutzen, während kontinuierlich laut gedacht wird — also die Gedanken zu verbalisieren, die aufkommen, während sich der Nutzer durch das \textit{User Interface} bewegt.\flqq\footnote{Nielsen, S. 196, 1993, [Accessed: 21.02.2017].}
Der \textit{Think Aloud}-Test ist ein qualitatives Testverfahren, das heißt es werden nur wenige und dafür ausgewählte Testprobanden benötigt.
\paragraph*{Testaufbau}Für den Test in der vorliegenden Arbeit wurden folgende Testprobanden ausgesucht:\begin{itemize}
	\item Zwei Testprobanden aus dem Bereich des Büromanagements
	\item Zwei Testprobanden aus dem Bereich der Gestaltung
	\item Zwei Testprobanden aus dem Bereich des Projektmanagements
	\item Zwei Testprobanden aus dem Bereich der Entwicklung
\end{itemize}
Die Testpersonen hatten keine bis wenig Vorkenntnisse im Umgang mit der \textit{Microsoft HoloLens} und es wurde darauf geachtet jeweils eine männliche und eine weibliche Testperson auszusuchen.
Als nächstes muss für Ruhe gesorgt werden, den Testern einzeln die Aufgabe aus Sektion~\ref{ssec:applicationScenario}, die es zu erfüllen gilt, erklärt werden sowie die Gesten einmalig vorgeführt werden.
%Darauf folgend schweigt der Autor und lässt die Testprobanden ihrer Aufgabe des \frqq Laut Denkens\flqq\ nachgehen.\newpage
\section{Auswertung}
Der Test aus Sektion \ref{sec:ThinkAloudTest}, welcher sich transkribiert im Anhang~\ref{transscrikpt} befindet, liefert im Kern die folgenden Ergebnisse:
\begin{table}[H]
	\centering
	\begin{tabular}{|l|p{8cm}|}
		\hline
		\textbf{Testperson} & männlich  \\
		\hline
		\textbf{Bereich} & Büromanagement \\
		\hline
		\textbf{Vorkenntnisse} & keine \\
		\hline
		\textbf{Eindruck} & Gut \\
		\hline
		\textbf{Hauptanmerkung} & Deutlicher machen, wie genau man den Roboter bewegt\\
		\hline
	\end{tabular}
	\caption{Testperson 1 -- Büromanagement, männlich.}
	\label{tab:scoreOffice1}
\end{table}

\begin{table}[H]
	\centering
	\begin{tabular}{|l|p{8cm}|}
		\hline
		\textbf{Testperson} & weiblich  \\
		\hline
		\textbf{Bereich} & Büromanagement \\
		\hline
		\textbf{Vorkenntnisse} & keine \\
		\hline
		\textbf{Eindruck} & schlecht \\
		\hline
		\textbf{Hauptanmerkung} & \textit{Cursor}-Hand-Koordination schwierig. Frustration, weil Geste nicht funktioniert\\
		\hline
	\end{tabular}
	\caption{Testperson 2 -- Büromanagement, weiblich.}
	\label{tab:scoreOffice2}
\end{table}
\begin{table}[H]
	\centering
	\begin{tabular}{|l|p{8cm}|}
		\hline
		\textbf{Testperson} & männlich  \\
		\hline
		\textbf{Bereich} & Gestaltung \\
		\hline
		\textbf{Vorkenntnisse} & keine \\
		\hline
		\textbf{Eindruck} & sehr gut \\
		\hline
		\textbf{Hauptanmerkung} & Wenig Latenz; funktioniert erstaunlich gut; \frqq wie Magie\flqq; Interface übersichtlich; \textit{Gaze} zusammen mit Hand schwierig\\
		\hline
	\end{tabular}
	\caption{Testperson 3 -- Gestaltung, männlich.}
	\label{tab:scoreDesign1}
\end{table}
\begin{table}[H]
	\centering
	\begin{tabular}{|l|p{8cm}|}
		\hline
		\textbf{Testperson} & weiblich  \\
		\hline
		\textbf{Bereich} & Gestaltung \\
		\hline
		\textbf{Vorkenntnisse} & keine \\
		\hline
		\textbf{Eindruck} & sehr gut \\
		\hline
		\textbf{Hauptanmerkung} & Schnelle Reaktion; funktioniert besser als gedacht; Interface geordnet\\
		\hline
	\end{tabular}
	\caption{Testperson 4 -- Gestaltung, weiblich.}
	\label{tab:scoreDesign2}
\end{table}
\begin{table}[H]
	\centering
	\begin{tabular}{|l|p{8cm}|}
		\hline
		\textbf{Testperson} & männlich  \\
		\hline
		\textbf{Bereich} & Projektmanagement \\
		\hline
		\textbf{Vorkenntnisse} & wenige Vorkenntnisse \\
		\hline
		\textbf{Eindruck} & sehr gut \\
		\hline
		\textbf{Hauptanmerkung} & Nicht intuitivste Steuerung; hatte angenommen das Interaktionsobjekt bildet absolute Positionswerte ab, nicht relative; wie Magie; Interface übersichtlich; \textit{Gaze} zusammen mit Hand schwierig\\
		\hline
	\end{tabular}
	\caption{Testperson 5 -- Projektmanagement, männlich.}
	\label{tab:scoreProjectmanagment1}
\end{table}
\begin{table}[H]
	\centering
	\begin{tabular}{|l|p{8cm}|}
		\hline
		\textbf{Testperson} & weiblich  \\
		\hline
		\textbf{Bereich} & Projektmanagement \\
		\hline
		\textbf{Vorkenntnisse} & keine Vorkenntnisse \\
		\hline
		\textbf{Eindruck} & sehr gut \\
		\hline
		\textbf{Hauptanmerkung} & Hatte Spaß bei der Bedienung; erstaunt, dass Roboter sich durch Geste bewegt hat; Interface ansprechend; Stimme leider nicht gut erkannt\\
		\hline
	\end{tabular}
	\caption{Testperson 6 -- Projektmanagement, weiblich.}
	\label{tab:scoreProjectmanagment2}
\end{table}

\begin{table}[H]
	\centering
	\begin{tabular}{|l|p{8cm}|}
		\hline
		\textbf{Testperson} & weiblich  \\
		\hline
		\textbf{Bereich} & Entwicklerin \\
		\hline
		\textbf{Vorkenntnisse} & keine Vorkenntnisse \\
		\hline
		\textbf{Eindruck} & sehr gut \\
		\hline
		\textbf{Hauptanmerkung} & \textit{HoloLens}-Gesten allgemein schwer und ungewohnt; Feedback für \textit{Tap-And-Hold}-Geste gewünscht; wenn Geste bekannt Aufgabe einfach zu bewältigen; eventuell Aufgabe in \textit{Interface} Schritt für Schritt abbilden\\
		\hline
	\end{tabular}
	\caption{Testperson 7 -- Entwicklung, weiblich.}
	\label{tab:scoreDevelopment1}
\end{table}

\begin{table}[H]
	
	\centering
	\begin{tabular}{|l|p{8cm}|}
		\hline
		\textbf{Testperson} & männlich  \\
		\hline
		\textbf{Bereich} & Entwickler \\
		\hline
		\textbf{Vorkenntnisse} & keine Vorkenntnisse \\
		\hline
		\textbf{Eindruck} & gut \\
		\hline
		\textbf{Hauptanmerkung} & Einfaches, gut verständliches Interface; Kontrollelement jedoch zu tief; Interaktionsobjekt auf Höhe des HoloInterface gewünscht\\
		\hline
	\end{tabular}
	\caption{Testperson 8 -- Entwicklung, männlich.}
	\label{tab:scoreDevelopment2}
\end{table}
Fasst man die Testergebnisse zusammen, lässt sich erschließen, dass bei technisch problemfreiem Ablauf, das heißt keine Blockade der Motoren auf Grund von Ungenauigkeiten im Tacho des \textit{Lego} Servomotors, die Testprobanden ihre Aufgabe gut erfüllen konnten. Manchen Testern musste mit erneuter Ansage der Instruktionen oder Hilfestellung die Gestensteuerung betreffend geholfen werden, sodass sie ihre Aufgabe erfüllen konnten. Heraus gestochen ist, dass zwei Tester die Bewegung des Interaktionsobjekts als absolute Position des Roboters auf der Schiene wahrgenommen haben. Außerdem wurde das \textit{User Interface} fast einstimmig als übersichtlich und angemessen beschrieben. Fünf der acht Testpersonen haben sich mehr eine optische Rückmeldung für die \textit{Tap-And-Hold}-Geste gewünscht, um in Erfahrung bringen zu können, ob das Interaktionselement erfolgreich \frqq angefasst\flqq\ wurde. Außerdem waren unverkennbar ein Staunen und Freude erkennbar, sobald der Roboter sich durch Gestik bedingt in Bewegung gesetzt hat. Ebenfalls erkennbar war, angefangen bei Testprobanden aus dem Bereich des Büromanagements, über Projektleitung und Gestaltung hin zum Bereich der Entwicklung, eine Zunahme der \textit{Usability}. 