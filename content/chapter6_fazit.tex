\chapter{Fazit und Ausblick}
Das Ergebnis hat gezeigt, dass es mit Technologien wie AR, Gesten- und Sprachsteuerung möglich ist, eine Maschine, das heißt einen Spielzeugroboter, zu teleoperieren und dass augmentierte \textit{User Interfaces} dabei hilfreich sind. Die Fertigstellung des funktionalen Prototypen wirft neue Fragestellungen auf. So stellt
sich beispielsweise die Frage, wie ein solches System am besten für den Betrieb mit echten Industriemaschinen und unter betrieblichen Bedingungen entworfen und implementiert sein soll.
Muss die Anwendung auf Seiten des Anwenders erheblich verbessert werden oder ist die Verbesserung der technischen, das heißt der \textit{Hardware}-Seite viel wichtiger und ein minimalistisches \textit{User Interface} vor den Augen des Anwenders reicht soweit aus? Wie schützt man eine solche Anwendung vor Fremdeinfluss und können Datenbrillen den Maschinenführern zusätzliche Sicherheit durch Sensorik bereitstellen? Wie stellt man sicher, dass das System generisch, mit einer Vielzahl von Robotern, funktioniert und ist eine Plattform mit der auf eine Vielzahl von Robotern zugegriffen werden kann sinnvoll oder gar gefährlich?
Die Verwendung von \textit{Unity}, speziell der Version 5.5f, der \textit{Microsoft HoloLens}, dem \textit{HoloToolkit} und \textit{MonoBrick} hat gezeigt, dass man mit einem solchen \textit{Framework} in der Lage ist, bei der Realisierung eines solchen Prototypen auf allen Ebenen die Programmiersprache \textit{C\#} verwendet werden kann auch wenn es mit Problemen der Kompatibilität von Schnittstellen einhergeht. Der Sachverhalt, dass sich viele der benutzten Technologien noch in den Kinderschuhen befinden, macht Hoffnung auf eine Umgebung, in der ohne größere Hindernisse und aus einer Hand, AR Anwendungen entwickelt werden können, und dass zukünftige Versionen von \textit{Unity} und der \textit{Microsoft HoloLens} dieses Vorhaben noch stärker vereinfachen und somit die AR Erfahrungen noch verbessern.
Allgemein lässt sich zusammenfassen, dass alle wesentlichen Ziele dieser Arbeit erreicht wurden und einer erfolgversprechenden Weiterentwicklung der Anwendung nichts im Wege steht, vor allem jetzt, da aufschlussreiche Ansatzpunkte aus dem \textit{Usability}-Test extrahiert werden konnten, welche die \textit{Usability} erneut verbessern.